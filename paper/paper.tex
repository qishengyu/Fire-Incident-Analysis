% Options for packages loaded elsewhere
\PassOptionsToPackage{unicode}{hyperref}
\PassOptionsToPackage{hyphens}{url}
\PassOptionsToPackage{dvipsnames,svgnames,x11names}{xcolor}
%
\documentclass[
  letterpaper,
  DIV=11,
  numbers=noendperiod]{scrartcl}

\usepackage{amsmath,amssymb}
\usepackage{lmodern}
\usepackage{iftex}
\ifPDFTeX
  \usepackage[T1]{fontenc}
  \usepackage[utf8]{inputenc}
  \usepackage{textcomp} % provide euro and other symbols
\else % if luatex or xetex
  \usepackage{unicode-math}
  \defaultfontfeatures{Scale=MatchLowercase}
  \defaultfontfeatures[\rmfamily]{Ligatures=TeX,Scale=1}
\fi
% Use upquote if available, for straight quotes in verbatim environments
\IfFileExists{upquote.sty}{\usepackage{upquote}}{}
\IfFileExists{microtype.sty}{% use microtype if available
  \usepackage[]{microtype}
  \UseMicrotypeSet[protrusion]{basicmath} % disable protrusion for tt fonts
}{}
\makeatletter
\@ifundefined{KOMAClassName}{% if non-KOMA class
  \IfFileExists{parskip.sty}{%
    \usepackage{parskip}
  }{% else
    \setlength{\parindent}{0pt}
    \setlength{\parskip}{6pt plus 2pt minus 1pt}}
}{% if KOMA class
  \KOMAoptions{parskip=half}}
\makeatother
\usepackage{xcolor}
\setlength{\emergencystretch}{3em} % prevent overfull lines
\setcounter{secnumdepth}{5}
% Make \paragraph and \subparagraph free-standing
\ifx\paragraph\undefined\else
  \let\oldparagraph\paragraph
  \renewcommand{\paragraph}[1]{\oldparagraph{#1}\mbox{}}
\fi
\ifx\subparagraph\undefined\else
  \let\oldsubparagraph\subparagraph
  \renewcommand{\subparagraph}[1]{\oldsubparagraph{#1}\mbox{}}
\fi

\usepackage{color}
\usepackage{fancyvrb}
\newcommand{\VerbBar}{|}
\newcommand{\VERB}{\Verb[commandchars=\\\{\}]}
\DefineVerbatimEnvironment{Highlighting}{Verbatim}{commandchars=\\\{\}}
% Add ',fontsize=\small' for more characters per line
\usepackage{framed}
\definecolor{shadecolor}{RGB}{241,243,245}
\newenvironment{Shaded}{\begin{snugshade}}{\end{snugshade}}
\newcommand{\AlertTok}[1]{\textcolor[rgb]{0.68,0.00,0.00}{#1}}
\newcommand{\AnnotationTok}[1]{\textcolor[rgb]{0.37,0.37,0.37}{#1}}
\newcommand{\AttributeTok}[1]{\textcolor[rgb]{0.40,0.45,0.13}{#1}}
\newcommand{\BaseNTok}[1]{\textcolor[rgb]{0.68,0.00,0.00}{#1}}
\newcommand{\BuiltInTok}[1]{\textcolor[rgb]{0.00,0.23,0.31}{#1}}
\newcommand{\CharTok}[1]{\textcolor[rgb]{0.13,0.47,0.30}{#1}}
\newcommand{\CommentTok}[1]{\textcolor[rgb]{0.37,0.37,0.37}{#1}}
\newcommand{\CommentVarTok}[1]{\textcolor[rgb]{0.37,0.37,0.37}{\textit{#1}}}
\newcommand{\ConstantTok}[1]{\textcolor[rgb]{0.56,0.35,0.01}{#1}}
\newcommand{\ControlFlowTok}[1]{\textcolor[rgb]{0.00,0.23,0.31}{#1}}
\newcommand{\DataTypeTok}[1]{\textcolor[rgb]{0.68,0.00,0.00}{#1}}
\newcommand{\DecValTok}[1]{\textcolor[rgb]{0.68,0.00,0.00}{#1}}
\newcommand{\DocumentationTok}[1]{\textcolor[rgb]{0.37,0.37,0.37}{\textit{#1}}}
\newcommand{\ErrorTok}[1]{\textcolor[rgb]{0.68,0.00,0.00}{#1}}
\newcommand{\ExtensionTok}[1]{\textcolor[rgb]{0.00,0.23,0.31}{#1}}
\newcommand{\FloatTok}[1]{\textcolor[rgb]{0.68,0.00,0.00}{#1}}
\newcommand{\FunctionTok}[1]{\textcolor[rgb]{0.28,0.35,0.67}{#1}}
\newcommand{\ImportTok}[1]{\textcolor[rgb]{0.00,0.46,0.62}{#1}}
\newcommand{\InformationTok}[1]{\textcolor[rgb]{0.37,0.37,0.37}{#1}}
\newcommand{\KeywordTok}[1]{\textcolor[rgb]{0.00,0.23,0.31}{#1}}
\newcommand{\NormalTok}[1]{\textcolor[rgb]{0.00,0.23,0.31}{#1}}
\newcommand{\OperatorTok}[1]{\textcolor[rgb]{0.37,0.37,0.37}{#1}}
\newcommand{\OtherTok}[1]{\textcolor[rgb]{0.00,0.23,0.31}{#1}}
\newcommand{\PreprocessorTok}[1]{\textcolor[rgb]{0.68,0.00,0.00}{#1}}
\newcommand{\RegionMarkerTok}[1]{\textcolor[rgb]{0.00,0.23,0.31}{#1}}
\newcommand{\SpecialCharTok}[1]{\textcolor[rgb]{0.37,0.37,0.37}{#1}}
\newcommand{\SpecialStringTok}[1]{\textcolor[rgb]{0.13,0.47,0.30}{#1}}
\newcommand{\StringTok}[1]{\textcolor[rgb]{0.13,0.47,0.30}{#1}}
\newcommand{\VariableTok}[1]{\textcolor[rgb]{0.07,0.07,0.07}{#1}}
\newcommand{\VerbatimStringTok}[1]{\textcolor[rgb]{0.13,0.47,0.30}{#1}}
\newcommand{\WarningTok}[1]{\textcolor[rgb]{0.37,0.37,0.37}{\textit{#1}}}

\providecommand{\tightlist}{%
  \setlength{\itemsep}{0pt}\setlength{\parskip}{0pt}}\usepackage{longtable,booktabs,array}
\usepackage{calc} % for calculating minipage widths
% Correct order of tables after \paragraph or \subparagraph
\usepackage{etoolbox}
\makeatletter
\patchcmd\longtable{\par}{\if@noskipsec\mbox{}\fi\par}{}{}
\makeatother
% Allow footnotes in longtable head/foot
\IfFileExists{footnotehyper.sty}{\usepackage{footnotehyper}}{\usepackage{footnote}}
\makesavenoteenv{longtable}
\usepackage{graphicx}
\makeatletter
\def\maxwidth{\ifdim\Gin@nat@width>\linewidth\linewidth\else\Gin@nat@width\fi}
\def\maxheight{\ifdim\Gin@nat@height>\textheight\textheight\else\Gin@nat@height\fi}
\makeatother
% Scale images if necessary, so that they will not overflow the page
% margins by default, and it is still possible to overwrite the defaults
% using explicit options in \includegraphics[width, height, ...]{}
\setkeys{Gin}{width=\maxwidth,height=\maxheight,keepaspectratio}
% Set default figure placement to htbp
\makeatletter
\def\fps@figure{htbp}
\makeatother

\KOMAoption{captions}{tableheading}
\makeatletter
\makeatother
\makeatletter
\makeatother
\makeatletter
\@ifpackageloaded{caption}{}{\usepackage{caption}}
\AtBeginDocument{%
\ifdefined\contentsname
  \renewcommand*\contentsname{Table of contents}
\else
  \newcommand\contentsname{Table of contents}
\fi
\ifdefined\listfigurename
  \renewcommand*\listfigurename{List of Figures}
\else
  \newcommand\listfigurename{List of Figures}
\fi
\ifdefined\listtablename
  \renewcommand*\listtablename{List of Tables}
\else
  \newcommand\listtablename{List of Tables}
\fi
\ifdefined\figurename
  \renewcommand*\figurename{Figure}
\else
  \newcommand\figurename{Figure}
\fi
\ifdefined\tablename
  \renewcommand*\tablename{Table}
\else
  \newcommand\tablename{Table}
\fi
}
\@ifpackageloaded{float}{}{\usepackage{float}}
\floatstyle{ruled}
\@ifundefined{c@chapter}{\newfloat{codelisting}{h}{lop}}{\newfloat{codelisting}{h}{lop}[chapter]}
\floatname{codelisting}{Listing}
\newcommand*\listoflistings{\listof{codelisting}{List of Listings}}
\makeatother
\makeatletter
\@ifpackageloaded{caption}{}{\usepackage{caption}}
\@ifpackageloaded{subcaption}{}{\usepackage{subcaption}}
\makeatother
\makeatletter
\@ifpackageloaded{tcolorbox}{}{\usepackage[many]{tcolorbox}}
\makeatother
\makeatletter
\@ifundefined{shadecolor}{\definecolor{shadecolor}{rgb}{.97, .97, .97}}
\makeatother
\makeatletter
\makeatother
\ifLuaTeX
  \usepackage{selnolig}  % disable illegal ligatures
\fi
\IfFileExists{bookmark.sty}{\usepackage{bookmark}}{\usepackage{hyperref}}
\IfFileExists{xurl.sty}{\usepackage{xurl}}{} % add URL line breaks if available
\urlstyle{same} % disable monospaced font for URLs
\hypersetup{
  pdftitle={Fire Incident Analysis},
  pdfauthor={Bruce Yu},
  colorlinks=true,
  linkcolor={blue},
  filecolor={Maroon},
  citecolor={Blue},
  urlcolor={Blue},
  pdfcreator={LaTeX via pandoc}}

\title{Fire Incident Analysis\thanks{Code and data are available at:
LINK.}}
\usepackage{etoolbox}
\makeatletter
\providecommand{\subtitle}[1]{% add subtitle to \maketitle
  \apptocmd{\@title}{\par {\large #1 \par}}{}{}
}
\makeatother
\subtitle{Exploring Toronto Fire Incident Data}
\author{Bruce Yu}
\date{September 24, 2024}

\begin{document}
\maketitle
\begin{abstract}
In this paper, we analyze fire incident data in Toronto to identify
patterns and relationships between response times, estimated losses, and
various incident characteristics. Using visualization techniques, we
provide insights into key metrics such as incident type, response time,
and resource allocation.
\end{abstract}
\ifdefined\Shaded\renewenvironment{Shaded}{\begin{tcolorbox}[breakable, enhanced, borderline west={3pt}{0pt}{shadecolor}, interior hidden, frame hidden, boxrule=0pt, sharp corners]}{\end{tcolorbox}}\fi

\hypertarget{introduction}{%
\section{Introduction}\label{introduction}}

In this paper, we analyze fire incident data in Toronto to gain insights
into the distribution of response times, estimated losses, and other key
variables. The data includes over 32,000 records detailing various
aspects of fire incidents such as incident type, number of responding
units, and the presence of fire alarm systems.

The remainder of this paper is structured as follows.
Section~\ref{sec-data} provides a detailed description of the data used.
In \textbf{?@sec-results}, we visualize the results, and
\textbf{?@sec-discussion} discusses the findings.

\hypertarget{sec-data}{%
\section{Data}\label{sec-data}}

The dataset used in this analysis contains over 32,000 records from
Toronto fire incidents. It includes various features such as response
times, estimated losses, number of responding units, presence of fire
alarms, and more. Each fire incident is categorized by incident type,
and there are records for both civilian and firefighter casualties.

The following sections explore different aspects of this dataset using
various visualization techniques.

\hypertarget{distribution-of-response-time}{%
\subsubsection{Distribution of Response
Time}\label{distribution-of-response-time}}

The distribution of response times for fire incidents is shown in
Figure~\ref{fig-response-time}. The majority of incidents have response
times between 4 and 6 minutes, with a few incidents having much longer
response times.

\begin{Shaded}
\begin{Highlighting}[]
\FunctionTok{ggplot}\NormalTok{(cleaned\_data, }\FunctionTok{aes}\NormalTok{(}\AttributeTok{x =}\NormalTok{ Response\_Time)) }\SpecialCharTok{+}
  \FunctionTok{geom\_histogram}\NormalTok{(}\AttributeTok{binwidth =} \DecValTok{1}\NormalTok{, }\AttributeTok{fill =} \StringTok{"skyblue"}\NormalTok{, }\AttributeTok{color =} \StringTok{"black"}\NormalTok{) }\SpecialCharTok{+}
  \FunctionTok{labs}\NormalTok{(}\AttributeTok{title =} \StringTok{"Distribution of Response Time"}\NormalTok{, }\AttributeTok{x =} \StringTok{"Response Time (minutes)"}\NormalTok{, }\AttributeTok{y =} \StringTok{"Count"}\NormalTok{) }\SpecialCharTok{+}
  \FunctionTok{theme\_minimal}\NormalTok{()}
\end{Highlighting}
\end{Shaded}

\begin{verbatim}
Warning: Removed 1395 rows containing non-finite outside the scale range
(`stat_bin()`).
\end{verbatim}

\begin{figure}[H]

{\centering \includegraphics{paper_files/figure-pdf/fig-response-time-1.pdf}

}

\caption{\label{fig-response-time}Distribution of Response Time}

\end{figure}

\hypertarget{distribution-of-estimated-loss}{%
\subsection{Distribution of Estimated
Loss}\label{distribution-of-estimated-loss}}

The distribution of estimated loss for fire incidents is highly skewed,
as shown in Figure~\ref{fig-estimated-loss}. Most incidents report
estimated losses below \$5,000, though a small number of incidents have
significantly higher losses, up to \$50,000 or more.

\begin{Shaded}
\begin{Highlighting}[]
\FunctionTok{ggplot}\NormalTok{(cleaned\_data, }\FunctionTok{aes}\NormalTok{(}\AttributeTok{x =}\NormalTok{ Estimated\_Loss)) }\SpecialCharTok{+}
  \FunctionTok{geom\_histogram}\NormalTok{(}\AttributeTok{binwidth =} \DecValTok{5000}\NormalTok{, }\AttributeTok{fill =} \StringTok{"lightgreen"}\NormalTok{, }\AttributeTok{color =} \StringTok{"black"}\NormalTok{) }\SpecialCharTok{+}
  \FunctionTok{labs}\NormalTok{(}\AttributeTok{title =} \StringTok{"Distribution of Estimated Loss"}\NormalTok{, }\AttributeTok{x =} \StringTok{"Estimated Loss ($)"}\NormalTok{, }\AttributeTok{y =} \StringTok{"Count"}\NormalTok{) }\SpecialCharTok{+}
  \FunctionTok{theme\_minimal}\NormalTok{()}
\end{Highlighting}
\end{Shaded}

\begin{verbatim}
Warning: Removed 5008 rows containing non-finite outside the scale range
(`stat_bin()`).
\end{verbatim}

\begin{figure}[H]

{\centering \includegraphics{paper_files/figure-pdf/fig-estimated-loss-1.pdf}

}

\caption{\label{fig-estimated-loss}Distribution of Estimated Loss}

\end{figure}

\hypertarget{response-time-by-incident-type}{%
\subsection{Response Time by Incident
Type}\label{response-time-by-incident-type}}

To examine how response times vary across different types of incidents,
we plotted a boxplot of response times grouped by incident type in
Figure~\ref{fig-response-incident-type}. It shows that the median
response time is relatively similar across most incident types, with a
few outliers where response times are much longer.

\begin{Shaded}
\begin{Highlighting}[]
\CommentTok{\# Simplify the Incident\_Type column}
\NormalTok{cleaned\_data }\OtherTok{\textless{}{-}}\NormalTok{ cleaned\_data }\SpecialCharTok{\%\textgreater{}\%}
  \FunctionTok{mutate}\NormalTok{(}\AttributeTok{Incident\_Type\_Simplified =} \FunctionTok{case\_when}\NormalTok{(}
    \FunctionTok{grepl}\NormalTok{(}\StringTok{"01"}\NormalTok{, Incident\_Type) }\SpecialCharTok{\textasciitilde{}} \StringTok{"Fire"}\NormalTok{,}
    \FunctionTok{grepl}\NormalTok{(}\StringTok{"02"}\NormalTok{, Incident\_Type) }\SpecialCharTok{\textasciitilde{}} \StringTok{"Explosion"}\NormalTok{,}
    \FunctionTok{grepl}\NormalTok{(}\StringTok{"03"}\NormalTok{, Incident\_Type) }\SpecialCharTok{\textasciitilde{}} \StringTok{"No Loss Outdoor Fire"}\NormalTok{,}
    \ConstantTok{TRUE} \SpecialCharTok{\textasciitilde{}} \StringTok{"Other"}
\NormalTok{  ))}

\CommentTok{\# Plot using the simplified names}
\FunctionTok{ggplot}\NormalTok{(cleaned\_data, }\FunctionTok{aes}\NormalTok{(}\AttributeTok{x =}\NormalTok{ Incident\_Type\_Simplified, }\AttributeTok{y =}\NormalTok{ Response\_Time)) }\SpecialCharTok{+}
  \FunctionTok{geom\_boxplot}\NormalTok{(}\AttributeTok{fill =} \StringTok{"lightblue"}\NormalTok{, }\AttributeTok{color =} \StringTok{"black"}\NormalTok{) }\SpecialCharTok{+}
  \FunctionTok{labs}\NormalTok{(}\AttributeTok{title =} \StringTok{"Response Time by Incident Type"}\NormalTok{, }\AttributeTok{x =} \StringTok{"Incident Type"}\NormalTok{, }\AttributeTok{y =} \StringTok{"Response Time (minutes)"}\NormalTok{) }\SpecialCharTok{+}
  \FunctionTok{theme\_minimal}\NormalTok{() }\SpecialCharTok{+}
  \FunctionTok{theme}\NormalTok{(}\AttributeTok{axis.text.x =} \FunctionTok{element\_text}\NormalTok{(}\AttributeTok{angle =} \DecValTok{90}\NormalTok{, }\AttributeTok{hjust =} \DecValTok{1}\NormalTok{))}
\end{Highlighting}
\end{Shaded}

\begin{verbatim}
Warning: Removed 1395 rows containing non-finite outside the scale range
(`stat_boxplot()`).
\end{verbatim}

\begin{figure}[H]

{\centering \includegraphics{paper_files/figure-pdf/fig-response-incident-type-1.pdf}

}

\caption{\label{fig-response-incident-type}Response Time by Incident
Type}

\end{figure}

\hypertarget{response-time-vs-estimated-loss}{%
\subsection{Response Time vs Estimated
Loss}\label{response-time-vs-estimated-loss}}

We also explored the relationship between response time and estimated
loss. The scatter plot in Figure~\ref{fig-response-estimated-loss}
suggests that there is a weak positive correlation between response time
and estimated loss. However, the trend is not strong, and there are a
large number of incidents with zero or low losses regardless of response
time.

\begin{Shaded}
\begin{Highlighting}[]
\FunctionTok{ggplot}\NormalTok{(cleaned\_data, }\FunctionTok{aes}\NormalTok{(}\AttributeTok{x =}\NormalTok{ Response\_Time, }\AttributeTok{y =}\NormalTok{ Estimated\_Loss)) }\SpecialCharTok{+}
  \FunctionTok{geom\_point}\NormalTok{(}\AttributeTok{alpha =} \FloatTok{0.5}\NormalTok{, }\AttributeTok{color =} \StringTok{"darkblue"}\NormalTok{) }\SpecialCharTok{+}
  \FunctionTok{geom\_smooth}\NormalTok{(}\AttributeTok{method =} \StringTok{"lm"}\NormalTok{, }\AttributeTok{color =} \StringTok{"red"}\NormalTok{, }\AttributeTok{se =} \ConstantTok{FALSE}\NormalTok{) }\SpecialCharTok{+}
  \FunctionTok{labs}\NormalTok{(}\AttributeTok{title =} \StringTok{"Response Time vs Estimated Loss"}\NormalTok{, }\AttributeTok{x =} \StringTok{"Response Time (minutes)"}\NormalTok{, }\AttributeTok{y =} \StringTok{"Estimated Loss ($)"}\NormalTok{) }\SpecialCharTok{+}
  \FunctionTok{theme\_minimal}\NormalTok{()}
\end{Highlighting}
\end{Shaded}

\begin{verbatim}
`geom_smooth()` using formula = 'y ~ x'
\end{verbatim}

\begin{verbatim}
Warning: Removed 6231 rows containing non-finite outside the scale range
(`stat_smooth()`).
\end{verbatim}

\begin{verbatim}
Warning: Removed 6231 rows containing missing values or values outside the scale range
(`geom_point()`).
\end{verbatim}

\begin{figure}[H]

{\centering \includegraphics{paper_files/figure-pdf/fig-response-estimated-loss-1.pdf}

}

\caption{\label{fig-response-estimated-loss}Response Time vs Estimated
Loss}

\end{figure}

\hypertarget{number-of-fire-incidents-by-station-area}{%
\subsection{Number of Fire Incidents by Station
Area}\label{number-of-fire-incidents-by-station-area}}

The number of fire incidents by station area is visualized in
Figure~\ref{fig-fire-incidents-station-area}. The bar plot shows the
number of fire incidents by station area, with the station areas
segmented into four distinct groups on the x-axis (around 100, 200, 300,
and 400). Each group represents different geographic regions or station
numbers in the dataset.

Station areas in the 300 range show the highest density of fire
incidents, with several exceeding 750 incidents. This might indicate
that these regions are more populated or prone to higher fire risks.
Other station areas (around 100, 200, and 400) also see fire incidents,
but their distribution is less dense compared to the 300 series. The
clustering of incidents around certain station areas could be driven by
multiple factors, such as population density, industrial areas, or
geographical layouts prone to fire hazards. Further investigation is
needed to determine whether these station areas correspond to specific
urban or suburban regions.

\begin{Shaded}
\begin{Highlighting}[]
\CommentTok{\# Remove rows with non{-}finite values in the Incident\_Station\_Area column}
\NormalTok{cleaned\_data\_filtered }\OtherTok{\textless{}{-}}\NormalTok{ cleaned\_data }\SpecialCharTok{\%\textgreater{}\%}
  \FunctionTok{filter}\NormalTok{(}\SpecialCharTok{!}\FunctionTok{is.na}\NormalTok{(Incident\_Station\_Area) }\SpecialCharTok{\&} \FunctionTok{is.finite}\NormalTok{(Incident\_Station\_Area))}

\CommentTok{\# Re{-}plot the data after filtering}
\FunctionTok{ggplot}\NormalTok{(cleaned\_data\_filtered, }\FunctionTok{aes}\NormalTok{(}\AttributeTok{x =}\NormalTok{ Incident\_Station\_Area)) }\SpecialCharTok{+}
  \FunctionTok{geom\_bar}\NormalTok{(}\AttributeTok{fill =} \StringTok{"orange"}\NormalTok{) }\SpecialCharTok{+}
  \FunctionTok{labs}\NormalTok{(}\AttributeTok{title =} \StringTok{"Number of Fire Incidents by Station Area"}\NormalTok{, }
       \AttributeTok{x =} \StringTok{"Station Area"}\NormalTok{, }
       \AttributeTok{y =} \StringTok{"Count"}\NormalTok{) }\SpecialCharTok{+}
  \FunctionTok{theme\_minimal}\NormalTok{() }\SpecialCharTok{+}
  \FunctionTok{theme}\NormalTok{(}\AttributeTok{axis.text.x =} \FunctionTok{element\_text}\NormalTok{(}\AttributeTok{angle =} \DecValTok{90}\NormalTok{, }\AttributeTok{hjust =} \DecValTok{1}\NormalTok{))}
\end{Highlighting}
\end{Shaded}

\begin{figure}[H]

{\centering \includegraphics{paper_files/figure-pdf/fig-fire-incidents-station-area-1.pdf}

}

\caption{\label{fig-fire-incidents-station-area}Number of Fire Incidents
by Station Area}

\end{figure}

\hypertarget{responding-units-vs-estimated-loss}{%
\subsection{Responding Units vs Estimated
Loss}\label{responding-units-vs-estimated-loss}}

The relationship between the number of responding units and estimated
loss is shown in Figure~\ref{fig-responding-units-loss}. There appears
to be a positive correlation between the number of units and the
estimated loss, which may indicate that larger incidents with higher
estimated losses require more resources.

\begin{Shaded}
\begin{Highlighting}[]
\FunctionTok{ggplot}\NormalTok{(cleaned\_data, }\FunctionTok{aes}\NormalTok{(}\AttributeTok{x =}\NormalTok{ Responding\_Units, }\AttributeTok{y =}\NormalTok{ Estimated\_Loss)) }\SpecialCharTok{+}
  \FunctionTok{geom\_point}\NormalTok{(}\AttributeTok{alpha =} \FloatTok{0.5}\NormalTok{, }\AttributeTok{color =} \StringTok{"purple"}\NormalTok{, }\AttributeTok{position =} \FunctionTok{position\_jitter}\NormalTok{(}\AttributeTok{width =} \FloatTok{0.2}\NormalTok{, }\AttributeTok{height =} \DecValTok{500}\NormalTok{)) }\SpecialCharTok{+}
  \FunctionTok{geom\_smooth}\NormalTok{(}\AttributeTok{method =} \StringTok{"lm"}\NormalTok{, }\AttributeTok{color =} \StringTok{"red"}\NormalTok{, }\AttributeTok{se =} \ConstantTok{FALSE}\NormalTok{) }\SpecialCharTok{+}
  \FunctionTok{labs}\NormalTok{(}\AttributeTok{title =} \StringTok{"Responding Units vs Estimated Loss"}\NormalTok{, }\AttributeTok{x =} \StringTok{"Number of Responding Units"}\NormalTok{, }\AttributeTok{y =} \StringTok{"Estimated Loss ($)"}\NormalTok{) }\SpecialCharTok{+}
  \FunctionTok{theme\_minimal}\NormalTok{() }\SpecialCharTok{+}
  \FunctionTok{xlim}\NormalTok{(}\DecValTok{0}\NormalTok{, }\DecValTok{50}\NormalTok{)}
\end{Highlighting}
\end{Shaded}

\begin{verbatim}
`geom_smooth()` using formula = 'y ~ x'
\end{verbatim}

\begin{verbatim}
Warning: Removed 5008 rows containing non-finite outside the scale range
(`stat_smooth()`).
\end{verbatim}

\begin{verbatim}
Warning: Removed 5009 rows containing missing values or values outside the scale range
(`geom_point()`).
\end{verbatim}

\begin{figure}[H]

{\centering \includegraphics{paper_files/figure-pdf/fig-responding-units-loss-1.pdf}

}

\caption{\label{fig-responding-units-loss}Responding Units vs Estimated
Loss}

\end{figure}

\hypertarget{references}{%
\section{References}\label{references}}



\end{document}
